\documentclass{jarticle}
\usepackage{amsmath,amssymb}
\usepackage{bm}


\begin{document}
\section{QR分解に基づく擬似逆行列}
縦長列フルランク行列$A\in \mathbb{R}^{n\times m}$が$A=QR$とQR分解されるとき、$A$の擬似逆行列$A^{+}$は
\begin{align}
A^{+} = R^{-1}Q^T
\end{align}
となる。このことを証明する。

まず、一般に、行列$A$に対して以下の4式を満たす行列$A^{+}$はただ一つ存在し、
この$A^{+}$がムーアペンローズの擬似逆行列であることが知られている。
\begin{align}
AA^{+}A = A
\\
A^{+}AA^{+} = A^{+}
\\
(AA^{+})^T = AA^{+}
\\
(A^{+}A)^T = A^{+}A
\end{align}
行列$B=R^{-1}Q^T$がこの式を満たすことを確認する。
$Q^TQ$は単位行列$I$に等しいので、
$BA=(R^{-1}Q^T)(QR)=I$であることがわかる。ゆえに、
\begin{align}
ABA = A(BA) = A
\\
BAB = (BA)B = B
\\
(BA)^T = I = BA
\end{align}
となる。
さらに
$AB=QRR^{-1}Q^T=QQ^T$は明らかに対称行列なので、
\begin{align}
(AB)^T = AB
\end{align}
以上より、$B$は$A$の擬似逆行列に等しい。
\end{document}
